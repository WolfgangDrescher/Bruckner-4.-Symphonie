\section{Vergleich der langsamen Sätze}

Der wohl grösste sichtbare Unterschied in der beiden Fassungen des langsamen Satzes ist die unterschiedliche Notation des \emph{B-Teils}.
In der Fassung von 1874 beginnnt der Satz und der Hauptsatz im \emph{Andante quasi allegretto}.
Der Seitensatz (\emph{B-Teil}) wird aber im Adagio notiert und in doppelt so schnellen Notenwerten.
Dadurch entsteht -- wenn das Adagio im Vergleich zum Andante halb so schnell gedacht ist --, dasselbe Metrum, aber wegen der langsameren Tempobezeichnung passt \emph{doppelt so viel Musik} in einen Takt.

Eine offene Frage bleibt, ob Bruckner wirklich mit Adagio tatsächlich ein halb so langsames Tempo gemeint hat wie das Andante, oder ob diese beiden Tempi nicht in einem strengen Verhältnis zueinander stehen und der Seitensatz, wie es zu der Zeit in der Aufführungspraxis üblich war, tatsächlich langsamer gedacht ist als der Hauptsatz.
Dagegen spricht für mich allerdings, dass Bruckner in der «Fassung 1878/80», also schon vier Jahre später, den Seitensatz so arrangiert hat, dass er auch im \emph{Andante quasi allegretto} klingt.

Hinrichsen schreibt dazu: «In der zweiten Fassung ist der Satz sogar um einen Takt länger, als in der ersten (247 anstatt 246 Takte); paradoxerweise wirkt er jedoch weitaus konziser und knapper»\autocite[76]{hinrichsen:bruckner}.

Das der langsame Satz in der «Fassung 1878/80» knapper klingt ist kein Paradoxon, sondern lässt sich sehr einfach nachvollziehen.
Wenn man in der «Fassung 1874» die Adagio-Takte in das Tempo Andante umschreiben würde, wie es Bruckner in der 2.~Fassung getan hat, wird man feststellen, dass die Erstfassung tätsächlich mehr Takte hat.
Rechnet man also die Adagio-Takte aus der Erstfassung doppelt, so würde dieser Satz 299 Takte umfassen und wäre damit ganze 52 Takte länger als die Zweitfassung.
Bei Tempo \quarternote~=~60 sind dass immerhin 3:29 Minuten mehr Musik!\footnote{Siehe Tabelle~\ref{tab:duration} auf Seite~\pageref{tab:duration}}

\begin{table}[htbp]
	\caption{Vergleich der Längen des langsamen Satzen aus Bruckners 4. Symphonie}
	\label{tab:duration}
	\centering
	\begin{tabular}{p{5.5cm}|cc}
		& «Fassung 1874» & «Fassung 1878/80» \\
		&& \\[-0.5em]
		\hline
		&& \\[-0.5em]
		Anzahl real notierter Takte & 246 & 247 \\
		&& \\[-0.5em]
		% TODO Zeile umbenennen?
		Anzahl fiktiver Takte \small{(Adagio angepasst wie in «Fassung 1878/80»)} & 299 & \emph{247} \\
		&& \\[-0.5em]
		Dauer \small{(bei \quarternote~=~60, ohne langsameres Tempo gegen Ende)} & 19:56 & 16:27 \\
	\end{tabular}
\end{table}

% TODO
Wo genau diese Takte eingespart wurden und möglichweise auch warum diese gekürzt wurden möchte ich im Folgenden aufzeigen.


\subsection{Kürzungen}

Betrachtet man die Gesamtform der beiden Sätze, so wird man feststellen, dass Bruckner in der zweiten Fassung (1878/80) die Absicht hatte, den langsamen Satz im Gegensatz zur Erstfassung so zu formen, dass er insgesamt eine stringentere Entwicklung bekommt.
In der späteren Fassung lässt Bruckner z.B. in der Reprise eine Wiederholung, bzw. einen Entwicklungsteil, mit Themenmaterial aus der Gesangsperiode komplett weg.
Dieser Abschnitt ab Studienziffer \studienziffer{K} (Takt 173) entfällt wahrscheinlich deswegen, weil er harmonisch in verschiedene Regionen abbiegt, aber diese jeweils sequenzartig fortschreiten und taktwesie auf scheinbar willkürlichen Stufen innehalten.
Lediglich eine Sequenz bei der der Bass schrittweise Terzenfällt und mit einer Synkopenkette abwechselnd Quintsextakkorde und Grundakkorde erklingen (quasi Quintfall) wird in der «Fassung 1878/80» als verweilendes Moment vor dem dritten Eintreten des Hauptsatzes (\emph{A-Teil}) erneut aufgegriffen\footnote{«Fassung 1878/80», Studienziffer \studienziffer{L}, Takt 187.}.
Allerdings in einer komplett anderen Instrumentation.

Bruckner bedient sich aber auch weiterer Mittel, als nur Weglassung ganzer Formteile, um den Satz insgesamt konziser und knapper zu bekommen.

% TODO subsubsection umbenennen
\subsubsection{Schnellere Übergänge zwischen den Formteilen}

\paragraph{Sequenz zwischen dem ersten Hauptthema und seiner Wiederholung}

Ganz am Anfang der Symphonie folgt nach dem Haupthema und der ersten zweitaktigen Ausweichung nach \emph{Ces-Dur} in der «Fassung 1874» eine Romanesca-Sequenz die aber ständig dem erwarteten Tongeschlecht ausweicht und nach vier Takten schliesslich plagal in \emph{C-Dur} kadenziert.
Nach einem Überleitungstakt erklingen wieder, aber in einem tieferern Register, die Einleitenden Takte (= T. 1 und 2) vor dem erneuten Eintritt des Hauptthemas in \emph{c-Moll}.
In der «Fassung 1878/80» hingegen verkürzt er die Sequenz um einen Takt, und schliess nach einem Nonvorhalt auch hier plagal und mit einer Takterstickung in \emph{c-Moll}.
Er kürzt an dieser Stelle also mit relativ einfachen Mitteln um vier Takte, die den Fluss des Stücks aber deutlich beschleunigen.


\subsubsection{Fehlende Sequenzen oder Sequenzglieder}

% Die aufwärts Sequenz bei Studienziffer \studienziffer{I} (Takt 164) über \emph{D-Dur}, \emph{Es-Dur} und \emph{E-Dur}

\paragraph{Kombinierte Sequenzglieder in der Schlussgruppe zwischen der Gesangsperiode und der Durchführung}

Eine besonders spannende Kürzung nimmt Bruckner kurz vor dem Übergang in die Durchfürhung vor.
In der Erstfassung komponierte Bruckner bei Buchstabe \studienziffer{D}\footnote{«Fassung 1874», Studienziffer \studienziffer{D}, Takt 70.} in der Überleitung zur Durchführung eine grösser angelegte Sequenz mit drei aufwärts steigenden Sequenzgliedern von jeweils sechs Takten.
Das erste Horn verbindet diese Sequenzglieder miteinander in dem es mit einem punktierten Rhythmus eine umspielte Dreiklangsbrechnung abwärts spielt und diese jeweils einen Ton höher ansetzt um in eine andere Tonartenregion zu kommen.
Darauf folgt jeweils eine Fauxbourdon-Sequenz mit einer \emph{7--6-consecutive}, die einmal komplett die Tonleiter abwärts schreitet und dann wieder in der Ausgangstonart kadenziert.
Das erste Sequenzglied ist in \emph{C}, das zweit in \emph{Cis} und das dritte in \emph{Dis}, bzw. notiert Bruckner enharmonisch verwechselt \emph{Es}.

In der «Fassung 1878/80» reduziert Bruckner diese Sequenz auf ein einziges Sequenzglied! In Takt 83 beginnt dieses mal die Flöte mit dem punktierten Motiv, das in der Erstfassung vom Horn gespielt wurde; in \emph{C}.
Direkt im Anschluss übernimmt das Horn aber dieses Motiv und imitiert es einen Ton höher in \emph{Des}.
Darauf folgt die Fauxbourdon-Sequenz von \emph{Des} (\emph{Cis}) aus gehend.
Die Sequenz bleibt aber nicht diatonisch in der Tonart, sondern bricht aus dem Fauxbourdon abrubt aus und kadenziert schliesslich in \emph{Es}.
Was in diesem Moment kompositorisch passiert, ist, dass erst ein komplettes Sequenzglied übersprungen wird anschliessend das zweite Sequenzglied mit dem dritten kombiniert wird und dadurch also erneut ein Sequenzglied übersprungen wird.
Von drei Sequenzgliedern bleibt also nur noch eins übrig.
Da es kurz, aber doch sehr deutlich \emph{harmonisch knirscht}, kann man hier besonders gut beobachten, dass Bruckner die Absicht hatte, den langsamen Satz insgesamt zu komprimieren und zu kürzen.


\subsection{Ergänzungen}

Ende Seitensatz wird in der frühen Fassung nur 3x wiederholt.

\subsection{Umarbeitungen}

Anfang.
Durchführung.
Zwölfachteltakt.

\subsubsection{Durchführung}

Lorem ipsum dolor sit amet, consectetur adipisicing elit, sed do eiusmod tempor incididunt ut labore et dolore magna aliqua. Ut enim ad minim veniam, quis nostrud exercitation ullamco laboris nisi ut aliquip ex ea commodo consequat. Duis aute irure dolor in reprehenderit in voluptate velit esse cillum dolore eu fugiat nulla pariatur. Excepteur sint occaecat cupidatat non proident, sunt in culpa qui officia deserunt mollit anim id est laborum.

\subsubsection{Variationen}

Lorem ipsum dolor sit amet, consectetur adipisicing elit, sed do eiusmod tempor incididunt ut labore et dolore magna aliqua. Ut enim ad minim veniam, quis nostrud exercitation ullamco laboris nisi ut aliquip ex ea commodo consequat. Duis aute irure dolor in reprehenderit in voluptate velit esse cillum dolore eu fugiat nulla pariatur. Excepteur sint occaecat cupidatat non proident, sunt in culpa qui officia deserunt mollit anim id est laborum.


\subsection{Form}

Lorem ipsum dolor sit amet, consectetur adipisicing elit, sed do eiusmod tempor incididunt ut labore et dolore magna aliqua. Ut enim ad minim veniam, quis nostrud exercitation ullamco laboris nisi ut aliquip ex ea commodo consequat. Duis aute irure dolor in reprehenderit in voluptate velit esse cillum dolore eu fugiat nulla pariatur. Excepteur sint occaecat cupidatat non proident, sunt in culpa qui officia deserunt mollit anim id est laborum.


\subsection{Harmonik}

Lorem ipsum dolor sit amet, consectetur adipisicing elit, sed do eiusmod tempor incididunt ut labore et dolore magna aliqua. Ut enim ad minim veniam, quis nostrud exercitation ullamco laboris nisi ut aliquip ex ea commodo consequat. Duis aute irure dolor in reprehenderit in voluptate velit esse cillum dolore eu fugiat nulla pariatur. Excepteur sint occaecat cupidatat non proident, sunt in culpa qui officia deserunt mollit anim id est laborum.

% TODO Es-Dur Teil?


\subsection{Instrumentation}

Lorem ipsum dolor sit amet, consectetur adipisicing elit, sed do eiusmod tempor incididunt ut labore et dolore magna aliqua. Ut enim ad minim veniam, quis nostrud exercitation ullamco laboris nisi ut aliquip ex ea commodo consequat. Duis aute irure dolor in reprehenderit in voluptate velit esse cillum dolore eu fugiat nulla pariatur. Excepteur sint occaecat cupidatat non proident, sunt in culpa qui officia deserunt mollit anim id est laborum.
