\section{Einleitung}

Im Musikwissenschaftlichen Seminar über Bruckners Symphonien bei Felix Diergarten in diesem Semester sind wir bei der 4. Symphonie über ein viel versprechendes Zitat gestossen.
Hans-Joachim Hinrichsen schreibt in seinem musikalischen Werkführer über Bruckners Sinfonien: «Insbesondere kann einen das genaue Studium der Umarbeitung des langsamen Satzes die philosophische Kunst des Staunens lehren»\autocite[76]{hinrichsen:bruckner}.
Er beschreibt im weiteren Verlauf ein paar Punkte, in denen sich die beiden Fassungen voneinander unterscheiden; wie z.B. die Tempoveränderung des B-Teils vom \emph{Adagio} zum \emph{Andante quasi Allegretto} in dem der Satz auch beginnt; oder auch einer großen Menge an Substanz die in den beiden B-Teilen weggenommen wurde.
Insgesamt schreibt Hinrichsen, dass durch die Umarbeitung des langsamen Satzs der 4. Symphonie ein «Dokument souveräner Redaktionsarbeit eines seiner Sache absolut sicheren Könners»\autocite[77]{hinrichsen:bruckner} sei.
Und: «Auf demselben formalen Grundriss ist im Rahmen derselben Anzahl von Takten und auf der Basis desselben thematischen Materials ein vollständig anderer Satz entstanden»\autocite[77]{hinrichsen:bruckner}.

% TODO
Mit dieser Ausgangslage habe ich eigene \emph{genaue Studien} des langsamen Satzes gemacht und dabei mit einem eher musiktheoretischen als musikwissenschatlichen Blickwinkel analysiert.
Ich hoffe damit im folgenden weitere Gründe für die «philosophische Kunst des Staunens» zeigen können.
