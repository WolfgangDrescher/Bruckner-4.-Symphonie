\section{Conclusio}

Zusammenfassend kann man sagen, dass Bruckner in der «Fassung 1878/80» des \emph{Andante quasi Allegretto} vier Jahr nach der Erstfassung von 1874 durch die Umarbeitung insgesamt mehr Farben und und auch formal konsequentere Zusammenhänge komponiert hat.
Gerade am Anfang kann man dies gut nachvollziehen mit dem neuen Charakter der einleitenden Takte oder der gekürzten Überleitung mittels Takterstickung zur Wiederholung des Hauptthemas.
Obwohl man beim Studium beider Fassungen in der Zweitfassung bemerken kann, dass Bruckner an einigen Stellen ausschweifendere Ideen und Sequenzen vorgesehen hatte, wirkt die Zweitfassung trotzdem insgesamt kompositorisch abgerundeter.
Nur in wenigen Momenten und nur im direkten Vergleich fallen Stellen auf die «gewaltsam» eingerichtet wurden um den Satz zu kürzen.
Und nur wenige Momente, wie z.B. den \emph{C-Dur}-Höhepunkt am Ende des Satzes, vermisse ich in der Zweitfassung.

Tatsächlich hat mich, um mit den Worten Hinrichsens zu sprechen, der langsame Satz die «Philosophie des Staunens gelehrt», aber weniger wie für ihn durch die Umarbeitung selbst, sondern viel eher durch ein intensives Partiturstudium des Satzes überhaupt; unabhängig von der Fassung.

Dass auf demselben formalen Grundriss und auf der Basis desselben thematischen Materials ein vollständig andere Satz entstanden ist, finde ich allerdings nicht.
Ein anderer Satz, ja, aber kein vollkommen anderer.
Für mich ist es eher eine Evolution als eine Revolution.
Was aber auf jeden Fall zutrifft, ist die Aussage Hinrichsens, dass die Umarbeitung des langsamen Satzes der 4. Symphonie ein «Dokument souveräner Redaktionsarbeit eines seiner Sache absolut sicheren Könners» sei.
