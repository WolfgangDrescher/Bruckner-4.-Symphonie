\section{Das Problem der Fassungen}

Wie ein Grossteil der Bruckner Symphonien gibt es auch bei der 4. Symphonie das Problem mit den unterschiedlichen Fassungen.
Ich möchte erst einen kurzen Überblick über die Problematik im Allgemeinen geben und anschliessend im konkreten Fall der 4. Symphonie erklären welche Fassungen es gibt und auf welche ich mich bei meinem Vergleich beziehe.


\subsection{Im allgemeinen zu den Symphonien}

Wären die Symphonien Bruckners alle von Anfang an ein Erfolg gewesen, würde es das Problem mit den unterschiedlichen Fassungen heute wohl nicht geben.
Dies sieht man z.B. daran, dass die 7. Symphonien, die bei der Uraufführung erfolgreich war, nur eine einzige Fassung aufweist.
Andere Symphonien wie die «Studiensymphonie» oder die «Annullierte», die von Bruckner nicht als \emph{aufführungswürdig} angesehen wurden, oder Symphonien die nicht zu Lebzeiten Bruckners aufgefüht wurden wie z.B. die 9. Symphonie sind ebenfalls befreit von dieser Problematik.
Es bleiben nur die ersten vier Symphonien und die 8. übrig, die aus unterschiedlichen Gründen überarbeitet wurden.

Dazu kommt, dass zusätzlich zu Bruckners eigenen Überarbeitungen selbst auch einige seiner Schüler, wie die Brüder Josef und Franz Schalk, Ferdinand Löwe und der Dirigent Felix Mottl Vorschläge für Korrekturen gaben und teilweise sogar direkte Umarbeitungen für Druckfassungen und Anpassungen für Aufführungen der Symphonien vornahmen.

Zu behaupten, dass Bruckner wegen seiner unsicheren Persönlichkeit und öffentlichen Misserfolgen seiner Werke unter \emph{Fremdeinfluss} geraten sei und er deswegen nur eingeschränkt als verantwortlicher Komponist der späteren Fassungen in Frage kommt ist somit auch nicht ganz korrekt.
Die Umarbeitungen seiner Schüler Schalk und Löwe wurden gründlich von Bruckner geprüft und korrigiert und dann von ihm selbst für den Druck autorisiert.
Hinrichsen schreibt: «Heute steht fest, dass solche Urteile nicht nur in höchstem Maße ungerecht sind, sondern auch die Verantwortung Bruckners für die Revisionen massiv unterschätzen.»\autocite[33]{hinrichsen:bruckner}

Prinzipiell gab es zwei grosse Überarbeitungsphasen: 1876-1880 und 1887-1891.
In der ersten dieser beiden Phasen korrigiert er einen Grossteil der vor 1876 entstandenen Symphonien, vorallem in metrischer Hinsicht.
Dies tut er unter anderem auch um der Theorie seines Lehrers Simon Sechters mit schweren und leichten Taken gerecht zu werden.
Vereinfacht gesagt wird jeder Takt von eins bis acht durchnummeriert und die musikalischen Phrasen- und Periodenlängen so angepasst, dass sie in dieses Schema passen.
In der zweiten Phase ab 1887 korrigiert er erneut viele seiner bereits abgeschlossenen Werke, da diese unter Mitwirkung seiner Schüler nun erstmals (bzw. erneut) in den Druck gehen sollen.


\subsection{Zu den Fassungen der 4. Symphonie}

Ende 1874 vollendet Bruckner vorläufig die Kompositionsarbeiten an seiner 4. Symphonie.
Neben mehreren Zwischenstufen lassen sich von dieser Symphonie zwei Gesamtfassungen und drei Fassungen des Finales voneinander unterscheiden.
Interessant ist, dass Bruckner noch vor der Uraufführung am 20. Februar 1881 in Wien die Symphonie mehrfach überarbeitete.
1878 nimmt er innerhalb der ersten grossen Überabrbeitungsphase Änderungen an der gesamten Symphonie vor.
% TODO Scherzo ?!
Es entsteht dabei die 2. Fassung mit dem sogenannten «Volksfest»-Finale.
Zwei Jahre später, immer noch vor der Uraufführung, komponierte er zu dieser Fassung ein komplett neues Finale, in der die Coda aus dem «Volksfest»-Finale an den Anfang des Satzes gestellt wird.
Als die Symphonie nach der Uraufführung zum ersten mal in Detuschland gespielt werden soll (und Bruckner damit zum ersten mal überhaupt in Detuschland gespielt wurde), empfiehlt der Dirigent Felix Mottl ihm, im neuen Finalsatz den Reprisenbeginn komplett zu streichen.
Als sich gegen Ende der 1880er Jahre abzeichnet, dass es zur Publikation dieser Symphonie kommen soll, entsteht eine komplette neue, 3., Fassung, deren Umarbeitung aber hauptsächlich durch den Brucknerschüler Ferdinand Löwe vorgenommen wurden, aber durch Bruckner selbst gründlich überprüft und autorisiert wurde.
Neben vielen Uminstrumentierungen in dieser Fassung (Becken, Piccoloflöte und Basstuba), wird ebenfalls der Vorschlag Mottls berücksichtigt, den Reprisenbeginn im Finalsatz komplett zu streichen.
Dies erforderte grössere Eingriffe in die harmonische Konzeption des Finalsatzes.
Weil die «Fassung 1888» als Druckausgabe erschien, ist diese Fassung für die Rezeptionsgeschichte von der grössten Bedeutung.

Zusammenfassend hier eine tabellarische Darstellung der wichtigsten Fassungen\autocite{wiki:bruckner4}:

\begin{itemize}
	\item \textbf{1874: 1. Fassung (= «Fassung 1874»)}
	\item 1878: 2. Fassung (mit dem sogenannte «Volksfest»-Finale)
	\item \textbf{1880: neues (3.) Finale für die 2. Fassung (= «Fassung 1878/80»)}
	% \item \textit{1881: Weggelassener Reprisenbeginn im Finale durch Felix Mottl, wie später in der «Fassung 1888»}
	% TODO Karlsruhe? (Mottl)
	% TODO New York?
	\item 1888: 3. Fassung (Finale: 4. Fassung, mit weggelassenem Reprisenbeginn), bearbeitet \textbf{von Ferdinand Löwe}, Druck: 1889 (= «Fassung 1888»)
\end{itemize}

Weil die Unterschiede im langsamen Satz der 4. Symphonie den grössten Sprung machen zwischen der «Fassung 1874» und der «Fassung 1878/80», und diese Fassungen von Bruckner alleine umgearbeitet wurden, ohne das Mitwirken seiner Schüler, habe ich mich entschieden für meinen Vergleich diese zwei Fassungen zu nehmen.

% TODO!
Aufgrund der oben genannten Bemerkungen von Hans-Joachim Hinrichsen werde ich bei meinen Analysen im Folgenden den Fokus also auf die oben bereits angesprochenen Punkte legen: instrumentatorische Anpassungen, Kürzungen - und welche Auswirkungen das auf die Formkonzeption und die Harmonik des langsamen Satzes hat. % TODO Mehr Punkte auflisten
