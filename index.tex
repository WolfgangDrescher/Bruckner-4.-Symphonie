\documentclass[a4paper]{scrartcl} % scrreprt

% XeLaTeX
% polyglossia
% fontspec

\usepackage[T1]{fontenc}
\usepackage[utf8]{inputenc}
\usepackage[nswissgerman]{babel}
% \usepackage[babel,german=swiss]{csquotes}
\usepackage[doublespacing]{setspace}
\usepackage{makeidx}
\usepackage{graphicx}
\usepackage[style=authortitle-ibid]{biblatex} % verbose-ibid
\usepackage[bottom,norule,hang,perpage]{footmisc}
\usepackage{geometry}
\usepackage{chngcntr}
\usepackage{textcomp}
% \usepackage{microtype}
\usepackage{wasysym}

\geometry{a4paper,top=30mm,right=27mm,bottom=30mm,left=27mm}
\setcounter{secnumdepth}{3}
\makeindex
\addbibresource{bibliography.bib}
\setlength{\footnotemargin}{2mm}
\renewcommand*{\labelnamepunct}{\addcolon\addspace} % Interpunktion in Zitaten zwischen Autor und Titel
% \renewcommand{\newunitpunct}{\addcomma\addspace} % Zitatelemente mit einem Komma trennen (nicht der standardmässig Punkt)
\renewcommand{\subtitlepunct}{\addperiod\addspace}
% \renewcommand{\footnoterule}{\rule{45mm}{0.1mm}{\vspace*{5mm}}} % Trennlinie über Footer
\counterwithin{figure}{section}

\title{Anton Bruckners 4. Symphonie in Es-Dur «Romantische», WAB 104}
\subtitle{Ein Vergleich des langsamen Satzes zwischen der «Fassung 1874» und «Fassung 1878/80»}
\subject{Mastervortrag}
\author{
	Wolfgang Drescher\thanks{Klasse: Prof. Dr. Felix Diergarten}\\
	Hochschule für Musik Freiburg
}
\date{4. Juli 2018}


\begin{document}

\maketitle

\tableofcontents
\clearpage

\section{Einleitung}

Im Musikwissenschaftlichen Seminar über Bruckners Symphonien bei Felix Diergarten in diesem Semester sind wir bei der 4. Symphonie über ein viel versprechendes Zitat gestossen.
Hans-Joachim Hinrichsen schreibt in seinem musikalischen Werkführer über Bruckners Sinfonien: «Insbesondere kann einen das genaue Studium der Umarbeitung des langsamen Satzes die philosophische Kunst des Staunens lehren»\autocite[76]{hinrichsen:bruckner}.
Er beschreibt im weiteren Verlauf ein paar Punkte, in denen sich die beiden Fassungen voneinander unterscheiden; wie z.B. die Tempoveränderung des B-Teils vom \emph{Adagio} zum \emph{Andante quasi Allegretto} in dem der Satz auch beginnt; oder auch einer großen Menge an Substanz die in den beiden B-Teilen weggenommen wurde.
Insgesamt schreibt Hinrichsen, dass durch die Umarbeitung des langsamen Satzs der 4. Symphonie ein «Dokument souveräner Redaktionsarbeit eines seiner Sache absolut sicheren Könners»\autocite[77]{hinrichsen:bruckner} sei.
Und: «Auf demselben formalen Grundriss ist im Rahmen derselben Anzahl von Takten und auf der Basis desselben thematischen Materials ein vollständig anderer Satz entstanden»\autocite[77]{hinrichsen:bruckner}.

% TODO
Mit dieser Ausgangslage habe ich eigene \emph{genaue Studien} des langsamen Satzes gemacht und dabei mit einem eher musiktheoretischen als musikwissenschatlichen Blickwinkel analysiert.
Ich hoffe damit im folgenden weitere Gründe für die «philosophische Kunst des Staunens» zeigen können.

\section{Das Problem der Fassungen}

Wie ein Grossteil der Bruckner Symphonien gibt es auch bei der 4. Symphonie das Problem mit den unterschiedlichen Fassungen.
Ich möchte erst einen kurzen Überblick über die Problematik im Allgemeinen geben und anschliessend im konkreten Fall der 4. Symphonie erklären welche Fassungen es gibt und auf welche ich mich bei meinem Vergleich beziehe.


\subsection{Im allgemeinen zu den Symphonien}

Wären die Symphonien Bruckners alle von Anfang an ein Erfolg gewesen, würde es das Problem mit den unterschiedlichen Fassungen heute wohl nicht geben.
Dies sieht man z.B. daran, dass die 7. Symphonien, die bei der Uraufführung erfolgreich war, nur eine einzige Fassung aufweist.
Andere Symphonien wie die «Studiensymphonie» oder die «Annullierte», die von Bruckner nicht als \emph{aufführungswürdig} angesehen wurden, oder Symphonien die nicht zu Lebzeiten Bruckners aufgefüht wurden wie z.B. die 9. Symphonie sind ebenfalls befreit von dieser Problematik.
Es bleiben nur die ersten vier Symphonien und die 8. übrig, die aus unterschiedlichen Gründen überarbeitet wurden.

Dazu kommt, dass zusätzlich zu Bruckners eigenen Überarbeitungen selbst auch einige seiner Schüler, wie die Brüder Josef und Franz Schalk, Ferdinand Löwe und der Dirigent Felix Mottl Vorschläge für Korrekturen gaben und teilweise sogar direkte Umarbeitungen für Druckfassungen und Anpassungen für Aufführungen der Symphonien vornahmen.

Zu behaupten, dass Bruckner wegen seiner unsicheren Persönlichkeit und öffentlichen Misserfolgen seiner Werke unter \emph{Fremdeinfluss} geraten sei und er deswegen nur eingeschränkt als verantwortlicher Komponist der späteren Fassungen in Frage kommt ist somit auch nicht ganz korrekt.
Die Umarbeitungen seiner Schüler Schalk und Löwe wurden gründlich von Bruckner geprüft und korrigiert und dann von ihm selbst für den Druck autorisiert.
Hinrichsen schreibt: «Heute steht fest, dass solche Urteile nicht nur in höchstem Maße ungerecht sind, sondern auch die Verantwortung Bruckners für die Revisionen massiv unterschätzen.»\autocite[33]{hinrichsen:bruckner}

Prinzipiell gab es zwei grosse Überarbeitungsphasen: 1876-1880 und 1887-1891.
In der ersten dieser beiden Phasen korrigiert er einen Grossteil der vor 1876 entstandenen Symphonien, vorallem in metrischer Hinsicht.
Dies tut er unter anderem auch um der Theorie seines Lehrers Simon Sechters mit schweren und leichten Taken gerecht zu werden.
Vereinfacht gesagt wird jeder Takt von eins bis acht durchnummeriert und die musikalischen Phrasen- und Periodenlängen so angepasst, dass sie in dieses Schema passen.
In der zweiten Phase ab 1887 korrigiert er erneut viele seiner bereits abgeschlossenen Werke, da diese unter Mitwirkung seiner Schüler nun erstmals (bzw. erneut) in den Druck gehen sollen.


\subsection{Zu den Fassungen der 4. Symphonie}

Ende 1874 vollendet Bruckner vorläufig die Kompositionsarbeiten an seiner 4. Symphonie.
Neben mehreren Zwischenstufen lassen sich von dieser Symphonie zwei Gesamtfassungen und drei Fassungen des Finales voneinander unterscheiden.
Interessant ist, dass Bruckner noch vor der Uraufführung am 20. Februar 1881 in Wien die Symphonie mehrfach überarbeitete.
1878 nimmt er innerhalb der ersten grossen Überabrbeitungsphase Änderungen an der gesamten Symphonie vor.
% TODO Scherzo ?!
Es entsteht dabei die 2. Fassung mit dem sogenannten «Volksfest»-Finale.
Zwei Jahre später, immer noch vor der Uraufführung, komponierte er zu dieser Fassung ein komplett neues Finale, in der die Coda aus dem «Volksfest»-Finale an den Anfang des Satzes gestellt wird.
Als die Symphonie nach der Uraufführung zum ersten mal in Detuschland gespielt werden soll (und Bruckner damit zum ersten mal überhaupt in Detuschland gespielt wurde), empfiehlt der Dirigent Felix Mottl ihm, im neuen Finalsatz den Reprisenbeginn komplett zu streichen.
Als sich gegen Ende der 1880er Jahre abzeichnet, dass es zur Publikation dieser Symphonie kommen soll, entsteht eine komplette neue, 3., Fassung, deren Umarbeitung aber hauptsächlich durch den Brucknerschüler Ferdinand Löwe vorgenommen wurden, aber durch Bruckner selbst gründlich überprüft und autorisiert wurde.
Neben vielen Uminstrumentierungen in dieser Fassung (Becken, Piccoloflöte und Basstuba), wird ebenfalls der Vorschlag Mottls berücksichtigt, den Reprisenbeginn im Finalsatz komplett zu streichen.
Dies erforderte grössere Eingriffe in die harmonische Konzeption des Finalsatzes.
Weil die «Fassung 1888» als Druckausgabe erschien, ist diese Fassung für die Rezeptionsgeschichte von der grössten Bedeutung.

Zusammenfassend hier eine tabellarische Darstellung der wichtigsten Fassungen\autocite{wiki:bruckner4}:

\begin{itemize}
	\item \textbf{1874: 1. Fassung (= «Fassung 1874»)}
	\item 1878: 2. Fassung (mit dem sogenannte «Volksfest»-Finale)
	\item \textbf{1880: neues (3.) Finale für die 2. Fassung (= «Fassung 1878/80»)}
	% \item \textit{1881: Weggelassener Reprisenbeginn im Finale durch Felix Mottl, wie später in der «Fassung 1888»}
	% TODO Karlsruhe? (Mottl)
	% TODO New York?
	\item 1888: 3. Fassung (Finale: 4. Fassung, mit weggelassenem Reprisenbeginn), bearbeitet \textbf{von Ferdinand Löwe}, Druck: 1889 (= «Fassung 1888»)
\end{itemize}

Weil die Unterschiede im langsamen Satz der 4. Symphonie den grössten Sprung machen zwischen der «Fassung 1874» und der «Fassung 1878/80», und diese Fassungen von Bruckner alleine umgearbeitet wurden, ohne das Mitwirken seiner Schüler, habe ich mich entschieden für meinen Vergleich diese zwei Fassungen zu nehmen.

% TODO!
Aufgrund der oben genannten Bemerkungen von Hans-Joachim Hinrichsen werde ich bei meinen Analysen im Folgenden den Fokus also auf die oben bereits angesprochenen Punkte legen: instrumentatorische Anpassungen, Kürzungen - und welche Auswirkungen das auf die Formkonzeption und die Harmonik des langsamen Satzes hat. % TODO Mehr Punkte auflisten

\section{Vergleich der langsamen Sätze}

Der wohl grösste sichtbare Unterschied in der beiden Fassungen des langsamen Satzes ist die unterschiedliche Notation des \emph{B-Teils}.
In der Fassung von 1874 beginnnt der Satz und der Hauptsatz im \emph{Andante quasi allegretto}.
Der Seitensatz (\emph{B-Teil}) wird aber im Adagio notiert und in doppelt so schnellen Notenwerten.
Dadurch entsteht -- wenn das Adagio im Vergleich zum Andante halb so schnell gedacht ist --, dasselbe Metrum, aber wegen der langsameren Tempobezeichnung passt \emph{doppelt so viel Musik} in einen Takt.

Eine offene Frage bleibt, ob Bruckner wirklich mit Adagio tatsächlich ein halb so langsames Tempo gemeint hat wie das Andante, oder ob diese beiden Tempi nicht in einem strengen Verhältnis zueinander stehen und der Seitensatz, wie es zu der Zeit in der Aufführungspraxis üblich war, tatsächlich langsamer gedacht ist als der Hauptsatz.
Dagegen spricht für mich allerdings, dass Bruckner in der «Fassung 1878/80», also schon vier Jahre später, den Seitensatz so arrangiert hat, dass er auch im \emph{Andante quasi allegretto} klingt.

Hinrichsen schreibt dazu: «In der zweiten Fassung ist der Satz sogar um einen Takt länger, als in der ersten (247 anstatt 246 Takte); paradoxerweise wirkt er jedoch weitaus konziser und knapper»\autocite[76]{hinrichsen:bruckner}.

Das der langsame Satz in der «Fassung 1878/80» knapper klingt ist kein Paradoxon, sondern lässt sich sehr einfach nachvollziehen.
Wenn man in der «Fassung 1874» die Adagio-Takte in das Tempo Andante umschreiben würde, wie es Bruckner in der 2.~Fassung getan hat, wird man feststellen, dass die Erstfassung tätsächlich mehr Takte hat.
Rechnet man also die Adagio-Takte aus der Erstfassung doppelt, so würde dieser Satz 299 Takte umfassen und wäre damit ganze 52 Takte länger als die Zweitfassung.
Bei Tempo \quarternote~=~60 sind dass immerhin 3:29 Minuten mehr Musik!\footnote{Siehe Tabelle~\ref{tab:duration} auf Seite~\pageref{tab:duration}}

\begin{table}[htbp]
	\caption{Vergleich der Längen des langsamen Satzen aus Bruckners 4. Symphonie}
	\label{tab:duration}
	\centering
	\begin{tabular}{p{5.5cm}|cc}
		& «Fassung 1874» & «Fassung 1878/80» \\
		&& \\[-0.5em]
		\hline
		&& \\[-0.5em]
		Anzahl real notierter Takte & 246 & 247 \\
		&& \\[-0.5em]
		% TODO Zeile umbenennen?
		Anzahl fiktiver Takte \small{(Adagio angepasst wie in «Fassung 1878/80»)} & 299 & \emph{247} \\
		&& \\[-0.5em]
		Dauer \small{(bei \quarternote~=~60, ohne langsameres Tempo gegen Ende)} & 19:56 & 16:27 \\
	\end{tabular}
\end{table}

% TODO
Wo genau diese Takte eingespart wurden und möglichweise auch warum diese gekürzt wurden möchte ich im Folgenden aufzeigen.

Text um neue\autocite{roeder:bruckner} Zitate\autocite{heinze:bruckner} zu testen.


\subsection{Kürzungen}

Lorem ipsum dolor sit amet, consectetur adipisicing elit, sed do eiusmod tempor incididunt ut labore et dolore magna aliqua. Ut enim ad minim veniam, quis nostrud exercitation ullamco laboris nisi ut aliquip ex ea commodo consequat. Duis aute irure dolor in reprehenderit in voluptate velit esse cillum dolore eu fugiat nulla pariatur. Excepteur sint occaecat cupidatat non proident, sunt in culpa qui officia deserunt mollit anim id est laborum.


\subsection{Form}

Lorem ipsum dolor sit amet, consectetur adipisicing elit, sed do eiusmod tempor incididunt ut labore et dolore magna aliqua. Ut enim ad minim veniam, quis nostrud exercitation ullamco laboris nisi ut aliquip ex ea commodo consequat. Duis aute irure dolor in reprehenderit in voluptate velit esse cillum dolore eu fugiat nulla pariatur. Excepteur sint occaecat cupidatat non proident, sunt in culpa qui officia deserunt mollit anim id est laborum.


\subsection{Harmonik}

Lorem ipsum dolor sit amet, consectetur adipisicing elit, sed do eiusmod tempor incididunt ut labore et dolore magna aliqua. Ut enim ad minim veniam, quis nostrud exercitation ullamco laboris nisi ut aliquip ex ea commodo consequat. Duis aute irure dolor in reprehenderit in voluptate velit esse cillum dolore eu fugiat nulla pariatur. Excepteur sint occaecat cupidatat non proident, sunt in culpa qui officia deserunt mollit anim id est laborum.


\subsection{Instrumentation}

Lorem ipsum dolor sit amet, consectetur adipisicing elit, sed do eiusmod tempor incididunt ut labore et dolore magna aliqua. Ut enim ad minim veniam, quis nostrud exercitation ullamco laboris nisi ut aliquip ex ea commodo consequat. Duis aute irure dolor in reprehenderit in voluptate velit esse cillum dolore eu fugiat nulla pariatur. Excepteur sint occaecat cupidatat non proident, sunt in culpa qui officia deserunt mollit anim id est laborum.

\section{Kontrapunktische Untersuchungen}

Lorem ipsum dolor sit amet, consectetur adipisicing elit, sed do eiusmod tempor incididunt ut labore et dolore magna aliqua. Ut enim ad minim veniam, quis nostrud exercitation ullamco laboris nisi ut aliquip ex ea commodo consequat. Duis aute irure dolor in reprehenderit in voluptate velit esse cillum dolore eu fugiat nulla pariatur. Excepteur sint occaecat cupidatat non proident, sunt in culpa qui officia deserunt mollit anim id est laborum.


\subsection{Trugschlüssige Wendungen}

Lorem ipsum dolor sit amet, consectetur adipisicing elit, sed do eiusmod tempor incididunt ut labore et dolore magna aliqua. Ut enim ad minim veniam, quis nostrud exercitation ullamco laboris nisi ut aliquip ex ea commodo consequat. Duis aute irure dolor in reprehenderit in voluptate velit esse cillum dolore eu fugiat nulla pariatur. Excepteur sint occaecat cupidatat non proident, sunt in culpa qui officia deserunt mollit anim id est laborum.


\subsection{Wechsel des Tongeschlechts}

Lorem ipsum dolor sit amet, consectetur adipisicing elit, sed do eiusmod tempor incididunt ut labore et dolore magna aliqua. Ut enim ad minim veniam, quis nostrud exercitation ullamco laboris nisi ut aliquip ex ea commodo consequat. Duis aute irure dolor in reprehenderit in voluptate velit esse cillum dolore eu fugiat nulla pariatur. Excepteur sint occaecat cupidatat non proident, sunt in culpa qui officia deserunt mollit anim id est laborum.

\subsection{Harmonik}

Ausweichung in der Harmonik des Hauptsatzes.

Lieblingstelle in der Durchführung der Erstfassung.

\section{Conclusio}

Lorem ipsum dolor sit amet, consectetur adipisicing elit, sed do eiusmod tempor incididunt ut labore et dolore magna aliqua. Ut enim ad minim veniam, quis nostrud exercitation ullamco laboris nisi ut aliquip ex ea commodo consequat. Duis aute irure dolor in reprehenderit in voluptate velit esse cillum dolore eu fugiat nulla pariatur. Excepteur sint occaecat cupidatat non proident, sunt in culpa qui officia deserunt mollit anim id est laborum.


\clearpage
\printbibliography

\end{document}
